\documentclass{article}
\usepackage{hyperref}

\title{AffordIt Algorithms Document}
\author{Tom Shulruff}
\date{\today}


\begin{document}
\maketitle

\section{Introduction}

In December of 2007, I sold a home in San Luis Obispo to relocate
to San Jose.  After renting an apartment for 6 months, I missed
the pleasure of owning a home (and the tax benefits).  So I started
shopping for another home.  But how much could I afford?

A simple equation answers this question; but it's complex
enough to make hand calculations a pain.  So I coded up a little
calculator. AffordIt is written in Java using Swing and the latest
 Java Run Time
Environment.  It's distributed with the intent that everybody 
should make this calculation to avoid getting in over their head with
a mortgage.

\section{The Basic Calculation}

Purchasing a home requires a down payment and, usually, a loan,
called a mortgage\footnote{
\href{http://www.freddiemac.com/corporate/buying_and_owning.html}{Freddie Mac} 
provides some excellent advice, and calculators  for
 prospective home buyers.
Of course, AffordIt is more thorough, accounting for HOA, taxes, and mortgage
 insurance.}.
Owning a home also requires purchasing property insurance 
and paying property taxes.  
If the loan is non-conforming\footnote{
A non-conforming loan is a loan that fails to meet bank criteria for 
funding.
These loans were very, very common in the housing bubble.},
then lenders will require mortgage insurance as well as normal
property insurance.
Owning a home
in a planned development (or a condominium) also requires paying
a monthly Home Owner's Association (HOA) fee. In some cases, the HOA 
dues include
property insurance.
So a typical monthly payment covers 
Principal, Interest, property Taxes, and Insurance (PITI).  
The interest and property tax payments are tax exempt.  

Lenders only allow debtors to spend a particular fraction of their
 gross income, usually about 35\%, on debt.
Lenders call this the debtor's Debt Ratio. All 
payments on debt (like PITI, car, or credit card payments) are
included in this fraction of a debtor's gross income.
Debtors have to take all of these variables into consideration when 
computing how much
 home they can afford. 
Using a years worth of 
payments at a time will keep things scaled well and simplify the 
computation.
Table \ref{table:variables} summarizes these variables.

\begin{table}
\begin{center}
\begin{tabular}{ll}
\hline
\hline
\multicolumn{2}{c}{\emph{Assets}} \\
$I$ & Gross Annual Income \\
$\rho$ & Debt Ratio \\
$D$ & Down Payment \\
$P$ & Mortgage Principal \\
$A$ & Available Income \\
$X$ & Market Moxie \\
\hline
\multicolumn{2}{c}{\emph{Liabilities}} \\
$H$ & Home Price \\
$i$ & Mortgage Annual Interest Rate (Percentage)\\
$m$ & Annualized Mortgage Payment \\
$s$ & Annual Insurance Premium \\
$q$ & Mortgage Insurance Rate (Percent of Principal)\\
$qP$ & Annual Mortgage Insurance Premium \\
$d$ & Annualized HOA Dues \\
$t$ & Property Tax Rate (Percent of Home Price) \\
$tH$ & Annual Property Taxes \\
$p$ & Annualized Other Debt Payment \\
\hline
\hline
\end{tabular}
\caption{The pertinent variables.}
\label{table:variables}
\end{center}
\end{table}

\pagebreak

Breaking down the expenses, the allowed
debt (gross income times debt ratio) has to cover 
a years worth of monthly mortgage payments (principal and interest),
 property taxes, 
annual insurance and HOA dues, mortgage insurance, and a year's
 worth of payments on 
any additional debt. In symbols:
\begin{equation}
\label{DebtRatio}
\rho I = m + tH + s + d + p + qP
\end{equation}

Uncle Sam wants you to own a home. To help with this, Uncle Sam 
made mortgage interest and property tax payments 
``income deductions.'' In other words, these bills get paid out 
of gross income, as opposed to net income. Post-tax money pays 
for insurance, other debt, and HOA dues. If available income is 
measured pre-tax, then we can scale these post-tax expenses up 
into pre-tax dollars to make a yet more conservative estimates.

Let's say the tax rate is about $40$\% overall. This means that 
post-tax dollars are worth about $60$\% of pre-tax dollars.
So a fraction of gross income has to cover
\begin{equation}
\label{AvailableIncomePostTax}
\rho I = m + tH + (s + d + p + qP) / 0.6
\end{equation}
This makes for a very conservative estimate (it 
leads to a smaller home price), but gives a better idea of what 
bills are really affordable by the end of the year.

\pagebreak

But that's only half of the answer. To get the other half, we first
 start with the 
price of the home as the sum of the mortgage principal and any
 down\footnote{
Loan and escrow fees are usually under 1\% of the purchase price of
 the home, so 
they're ignored here. Otherwise, they'd subtract out of the down
 payment.} payment:
\begin{equation}
\label{HomePrice}
H = P + D
\end{equation}
A year's worth of payments\footnote{From Wolfram Research's 
\href{http://mathworld.wolfram.com/Mortgage.html}{\emph{MathWorld}} 
and scaled to 12 months.} on
a principal with a 30-year fixed\footnote{
There are other payment plans. The 30-year fixed mortgage has the
 highest monthly payment. This makes the most conservative limit 
on the maximum affordable price for a home.} annual percentage 
rate is
\begin{equation}
\label{MonthlyPayment}
m = Pi \left( \frac{((i/12) + 1)^{360}}{((i/12) + 1)^{360} - 1} \right)
\end{equation}
Substituting equations (\ref{HomePrice}) and (\ref{MonthlyPayment}) into 
equation (\ref{AvailableIncomePostTax}) gives:
\begin{equation}
\label{TempPrincipal}
\rho I = P i \left( \frac{((i/12) + 1)^{360}}{((i/12) + 1)^{360} - 1} \right)
+ t(D + P) + (s + d + p + qP) / 0.6
\end{equation}
Rearranging equation (\ref{TempPrincipal}) gives the principal ($P$) that 
this debt ratio can afford to carry with all the payments included:
\begin{equation}
\label{FinalPrincipal}
P = \frac{\rho I - tD - (s + d + p) / 0.6}{t + (q / 0.6) +
i \left( \frac{((i/12) + 1)^{360}}{((i/12) + 1)^{360} - 1} \right)}
\end{equation}
Finally, substituting equation (\ref{FinalPrincipal}) back into equation
 (\ref{HomePrice}) gives the maximum affordable price a person can pay 
for a home:
\begin{equation}
\label{FinalHomePrice}
H = \frac{\rho I - tD - (s + d + p) / 0.6}{t + (q / 0.6) +
i \left( \frac{((i/12) + 1)^{360}}{((i/12) + 1)^{360} - 1} \right)} + D
\end{equation}

Let's introduce two new variables, representing annual available income
 and a scaling factor to make this a little more palatable. Let the 
annual available income be the numerator of equation (\ref{FinalPrincipal})
\begin{equation}
\label{AvailableIncome}
A = \rho I - tD - (s + d + p) / 0.6
\end{equation}
The rest of equation (\ref{FinalPrincipal}) represents the amount of
 principal that each dollar of available income purchases 
\begin{equation}
\label{Moxie}
X = \frac{1}{t + (q/0.6) +
i \left( \frac{((i/12) + 1)^{360}}{((i/12) + 1)^{360} - 1} \right)}
\end{equation}
This scaling factor\footnote{ 
Moxie is in units of time (years, in this case). Maybe time really is
 money...}
will prove useful in the hunt for a home. For kicks, let's call this
 the market's ``Moxie." Putting it all together gives a simplified
 equation for the maximum price a person can afford to pay for a home:
\begin{equation}
\label{SimplifiedHomePrice}
H = AX + D
\end{equation}

\pagebreak


\section{Using Your Moxie}

Knowing the amount of debt a dollar can service can be very, 
very useful when 
shopping for a home.  Let's say you make \$$100,000$ a year, have
 \$$150,000$ for down payment, and secured a mortgage rate of 
$6.25$\%.  This makes for a Moxie of about $11.5$ years. 
In California, property taxes are a percentage of the
 selling price of the home, typically $1.3$\%.  Fire and earthquake
 insurance on a home is about \$$1000$ a year. If a lender allows
 $35$\% debt ratio, you can afford a \$$511,201$ home.

Of course, that was a very idealized example. Most people don't have 
\$$150,000$ laying around for a down payment on a house. If you have
less than $20$\% of the price of the home available for a down payment,
then you have to get mortgage insurance. Mortgage 
insurance is usually
between $1.5$\% and $6.0$\% of the 
principal\footnote{According to 
\href{http://en.wikipedia.org/wiki/Mortgage_insurance}{Wikipedia}.} 
 annually. As shown in Equation (\ref{Moxie}), mortgage insurance 
reduces Moxie. For example, if you only have \$$50,000$ for a down
payment, but can find a low mortgage insurance rate of $1.5$\%, Moxie
is reduced to $8.94$ years and you can only afford a \$$342,112$ house.
By adding in mortgage insurance, you've incurred a \$$69,089$ penalty
to affording a house.

In the San Francisco Bay Area, there are no single family homes
 that cheap (that you might want to actually occupy).
  There are, however, a variety of condominiums within this 
price range.  All condominiums require HOA dues.  With some 
algebra, the HOA dues can be said to add to the price of the home:
\begin{equation}
H = H_{condo} + dX
\end{equation}
HOA dues take away from annual available income.  Effectively, 
they add to the cost of the home.  In this way, we capitalize 
HOA dues.

Let's say there's a condominium available for \$$325,000$ with 
HOA dues of \$$350$ a month (\$$4,200$ annually).  Keeping the 
same mortgage rate and property tax rate that produced a Moxie of
 about $8.94$ years.  This puts the effective home price 
at \$$362,548$ - too expensive.

Similarly, if you're paying for a car at \$$350$ a month and 
owe less than \$$37,584$, you should pay off the automotive debt 
with mortgage debt. Removing that payment will allow you to 
afford the additional principal.

\pagebreak

\section{The Final Budget}

With all this in mind, let's take a look at what the monthly 
budget looks like for a buyer making \$$85,000$ who's lucky 
enough to have \$$50,000$ for a down payment. Table 
\ref{table:budget} itemizes a typical budget.
\begin{table}
\begin{center}
\begin{tabular}{r|l}
\hline
\hline
\$$85,000.00$ & Gross annual income \\
$35$\% & Debt Ratio \\
\$$2479.17$ & Allowed Monthly Payments \\
\hline
(\$$1555.53$) & Monthly Principal \& Interest payment \\
(\$$368.09$) & Monthly Property Taxes \\
(\$$83.33$) & Monthly Insurance Payment \\
(\$$250.00$) & Monthly HOA Payment \\
(\$$2,256.95$) & Total Monthly Payments \\
\hline
\$$51,000.00$ & Net annual income after 60\% taxes \\
\$$4,250.00$ & Net monthly income after taxes \\
\$$1,993.05$ & Money Left After Taxes For Other Stuff \\
\hline
\hline
\end{tabular}
\caption{The final budget.}
\label{table:budget}
\end{center}
\end{table}
Thanks to the fact that we scaled the HOA and insurance 
payment up into pre-tax dollars before making the estimate, 
the bills come in under the $35$\% debt ratio.

\section{Summary}

I hope this tool is as useful to you as it has been for me.  
It's been fun to learn Java and provide this tool as a free 
application.

\end{document}
