\documentclass{article}
\usepackage{hyperref}
\hypersetup{colorlinks = true,
            linkcolor = black,
            urlcolor = blue}

\title{AffordIt Algorithms Document}
\author{Tom Shulruff}
\date{\today}


\begin{document}
\maketitle

\section{Introduction}

A couple of years ago, I sold a home in San Luis Obispo to relocate
to San Jose.  Since then, I've missed
the pleasure -- and tax benefits -- of owning a home.  So I started
shopping for another house.  But how much can I afford?

With my Bachelor's of Science in Math and Physics in hand, 
I derived a simple equation to answer this question.
It's simple, but it's complex
enough to make hand calculations a pain.  So I created a little
calculator. 

It's distributed with the intent that everybody 
should make this calculation to avoid getting in over their 
head with a mortgage.

AffordIt is written in Java using Swing and the latest
 Java Run Time Environment.  

\section{The Basic Calculation}

To purchase a home you need a down payment and, usually, a loan
called a mortgage\footnote{
\href{http://www.freddiemac.com}{Freddie Mac} 
provides some excellent 
\href{http://www.freddiemac.com/corporate/buying_and_owning.html}{advice}, 
and 
\href{http://www.freddiemac.com/corporate/buyown/english/calcs_tools/}{calculators}  
for prospective home buyers.
Of course, AffordIt is more thorough, accounting for HOA, taxes, and mortgage
 insurance.}.
You also need to get property insurance 
and pay property taxes.  
If the loan is non-conforming\footnote{
A non-conforming loan is a loan that fails to meet bank criteria for 
funding.
These loans were very, very common in the housing bubble.},
then lenders will require you to get mortgage insurance in addition to 
property insurance. If you want to live in a home
or a condo in a planned development you will also need to pay
a monthly Home Owner's Association (HOA) fee. In some cases, the HOA 
dues include
property insurance.
A typical monthly payment covers 
principal, interest, property taxes, and insurance (PITI).  

Lenders only allow debtors to spend a particular fraction of their
 gross income, usually about 35\%, on housing or other debt.
Lenders call this the debtor's ``debt ratio.'' 

You need to take all of these variables into consideration while 
computing how much  home you can afford. 
Table \ref{table:variables} summarizes these variables.

\begin{table}
\begin{center}
\begin{tabular}{ll}
\multicolumn{2}{c}{Let's take a look at what we're working with:} \\
\hline
\hline
\multicolumn{2}{c}{\emph{Assets}} \\
$I$ & Gross Annual Income \\
$\rho$ & Debt Ratio \\
$D$ & Down Payment \\
$P$ & Mortgage Principal \\
$A$ & Available Income ($\rho I$) \\
$X$ & Market Moxie \emph{(see below)} \\
\hline
\multicolumn{2}{c}{\emph{Liabilities}} \\
$H$ & Home Price \\
$i$ & Mortgage Annual Interest Rate (Percentage)\\
$m$ & Annualized Mortgage Payment \\
$s$ & Annual Insurance Premium \\
$q$ & Mortgage Insurance Rate (Percent of Principal)\\
$qP$ & Annual Mortgage Insurance Premium \\
$d$ & Annualized HOA Dues \\
$t$ & Property Tax Rate (Percent of Home Price) \\
$tH$ & Annual Property Taxes \\
$p$ & Annualized Other Debt Payment \\
\hline
\hline
\end{tabular}
\caption{The pertinent variables.}
\label{table:variables}
\end{center}
\end{table}

Breaking down the expenses: the allowed
debt (gross income times debt ratio) has to cover 
a years worth of monthly mortgage payments (principal and interest),
 property taxes, insurance dues, HOA dues, garden gnomes,
mortgage insurance dues, and  
any additional debt payments. In symbols:
\begin{equation}
\label{DebtRatio}
\rho I = m + tH + s + d + p + qP
\end{equation}
Simplify the computation by using a year's worth of 
payments at a time.

Uncle Sam wants you to own a home. To help with this, the government 
made mortgage interest and property tax payments 
``income deductions.''

 In other words, mortgage interest and property taxes get paid out 
of gross income. Whereas
insurance payments, HOA dues, and any other debt payments 
are paid after taxes have been deducted from your income. 

If available income is 
measured pre-tax, then we can scale these post-tax expenses up 
into pre-tax dollars to make a yet more conservative estimate.

Let's say the tax rate is about $40$\% overall. This means that 
post-tax dollars are worth about $60$\% of pre-tax dollars.

So a fraction of gross income has to cover the post-tax costs, 
which now looks like
\begin{equation}
\label{AvailableIncomePostTax}
\rho I = m + tH + (s + d + p + qP) / 0.6
\end{equation}
This makes for a very conservative estimate (it 
leads to a more frugal home), but gives a better idea of what 
bills are really affordable by the end of the year.

\pagebreak

But that's only half of the answer. To get the other half, we 
 start with the 
price of the house as the sum of the mortgage principal and any
 down\footnote{
Loan and escrow fees are usually under 1\% of the purchase price of
 the home, so 
they're ignored here. Otherwise, they'd subtract out of the down
 payment.} payment:
\begin{equation}
\label{HomePrice}
H = P + D
\end{equation}
A year's worth of payments\footnote{
\href{http://mathworld.wolfram.com/topics/Pegg.html}{Pegg, Ed Jr.} "Mortgage." From 
\href{http://mathworld.wolfram.com/}{MathWorld} -- A Wolfram Web Resource, created by 
\href{http://mathworld.wolfram.com/about/author.html}{Eric W. Weisstein}. 
\url{http://mathworld.wolfram.com/Mortgage.html}}
 on a principal with a 30-year fixed\footnote{
There are other payment plans. This one is common and allows
for a simple limit on the maximum affordable price for a home.} 
annual interest rate is:
\begin{equation}
\label{MonthlyPayment}
m = Pi \left( \frac{((i/12) + 1)^{360}}{((i/12) + 1)^{360} - 1} \right)
\end{equation}
Substituting equations (\ref{HomePrice}) and (\ref{MonthlyPayment}) into 
equation (\ref{AvailableIncomePostTax}) gives:
\begin{equation}
\label{TempPrincipal}
\rho I = P i \left( \frac{((i/12) + 1)^{360}}{((i/12) + 1)^{360} - 1} \right)
+ t(D + P) + (s + d + p + qP) / 0.6
\end{equation}
Rearranging equation (\ref{TempPrincipal}) gives the principal ($P$) that 
this debt ratio can afford to carry with all the payments included:
\begin{equation}
\label{FinalPrincipal}
P = \frac{\rho I - tD - (s + d + p) / 0.6}{t + (q / 0.6) +
i \left( \frac{((i/12) + 1)^{360}}{((i/12) + 1)^{360} - 1} \right)}
\end{equation}
Finally, substituting equation (\ref{FinalPrincipal}) back into equation
 (\ref{HomePrice}) gives the maximum affordable price a person can pay 
for a home:
\begin{equation}
\label{FinalHomePrice}
H = \frac{\rho I - tD - (s + d + p) / 0.6}{t + (q / 0.6) +
i \left( \frac{((i/12) + 1)^{360}}{((i/12) + 1)^{360} - 1} \right)} + D
\end{equation}

Let's introduce two new variables to make this a 
little more palatable. One will represent
 annual available income
 and is the numerator of equation (\ref{FinalPrincipal}):
\begin{equation}
\label{AvailableIncome}
A = \rho I - tD - (s + d + p) / 0.6
\end{equation}
The rest of equation (\ref{FinalPrincipal}) represents the amount of
 principal that each dollar of available income can purchase.
For kicks, let's call this x-factor ``Moxie'': 
\begin{equation}
\label{Moxie}
X = \frac{1}{t + (q/0.6) +
i \left( \frac{((i/12) + 1)^{360}}{((i/12) + 1)^{360} - 1} \right)}
\end{equation}
Annual available income is measured in dollars per year.
To get final value of dollars for the price of a house, Moxie
must be measured in years. Maybe time really is money...

Moxie will prove useful in the hunt for a home. 
Putting it all together gives a simplified
 equation for the maximum price someone can afford to pay for a home:
\begin{equation}
\label{SimplifiedHomePrice}
H = AX + D
\end{equation}

\pagebreak


\section{Using Your Moxie}

Knowing the amount of debt a dollar can buy you can be very, 
very useful when 
shopping for a home.  Let's say you make \$$100,000$ a year, have
 \$$150,000$ for down payment, and secured a mortgage rate of 
$6.25$\%.  In this case, your Moxie is about $11.5$ years. 

In California, property taxes are a percentage of the
 selling price of the home, typically $1.3$\%.  Fire and earthquake
 insurance on a home is about \$$1000$ a year. If your bank allows 
you a $35$\% debt ratio, you can afford a \$$511,201$ home.

\subsection{Private Mortgage Insurance}

Of course, that was a very idealized example. Most people don't have 
\$$150,000$ laying around for a down payment on a house. 

If you have
less than $20$\% of the price of the home available for a down payment,
then you have to get mortgage insurance. Mortgage 
insurance is usually
between $1.5$\% and $6.0$\% of the 
principal\footnote{According to the 
\href{http://en.wikipedia.org/wiki/Main_Page}{Wikipedia}
article on 
\href{http://en.wikipedia.org/wiki/Mortgage_insurance}{Mortgage Insurance}.} 
 annually. 

As shown in Equation (\ref{Moxie}), mortgage insurance 
reduces Moxie. For example, if you only have \$$50,000$ for a down
payment, and you can find a low mortgage insurance rate of $1.5$\%,
your Moxieis reduced to $8.94$ years and you can only 
afford a \$$342,112$ house.

Needing mortgage insurance means you've incurred a \$$69,089$ penalty
to the maximum amount you can afford to pay for a house. 
If you have bad credit, then the mortgage
insurance rate can get up to $6.0$\%, dropping Moxie down to $5.35$
years and the maximum home price down to \$$224,883$. 

Times have
have really changed since the housing bubble burst.

\subsection{Home Owners Associations}

In the San Francisco Bay Area, there are no single family homes
 that cheap (that you might want to actually occupy).

  There are, however, a variety of condominiums within this 
price range.  All condos require HOA dues.  

With some 
algebra, you can find out what HOA dues add to the price of a home:
\begin{equation}
H_{equiv} = H_{condo} + dX
\end{equation}

HOA dues take away from annual available income.  Effectively, 
they add to the cost of the home.  In this way, we capitalize 
HOA dues into a ``house equivalent price.''

Let's say there's a condominium available for \$$325,000$ with 
HOA dues of \$$350$ a month (\$$4,200$ annually).  Using the
same rates that produced a Moxie of about $8.94$ years, 
the ``house equivalent price'' is  \$$362,548$ - too expensive.

Similarly, if you're paying for a car at \$$350$ a month and 
owe less than \$$37,584$, you should pay off the automotive debt 
with mortgage debt. Remove that car payment, and you can  
afford to borrow the additional principal.

\section{The Final Budget}

With all this in mind, let's take a look at what the monthly 
budget looks like for a buyer making \$$85,000$ and is lucky 
enough to have \$$50,000$ for a down payment. 

According to 
the equations, this lucky buyer can afford to pay \$$286,386$ 
for house, 
but will need mortgage insurance in order to get the loan.

Table 
\ref{table:budget} itemizes a typical budget.
\begin{table}
\begin{center}
\begin{tabular}{r|l}
\multicolumn{2}{c}{Here's the numbers for a typical home buyer:} \\
\hline
\hline
\$$85,000.00$ & Gross Annual Income \\
$35$\% & Debt Ratio \\
\$$2479.17$ & Allowed Monthly Payments \\
\hline
(\$$1455.48$) & Monthly Principal \& Interest Payment \\
(\$$310.26$) & Monthly Property Taxes \\
(\$$83.33$) & Monthly Insurance Payment \\
(\$$417.65$) & Monthly Mortgage Insurance Payment \\
(\$$2,266.95$) & Total Monthly Payments \\
\hline
\$$51,000.00$ & Net Annual Income (after taxes) \\
\$$4,250.00$ & Net Monthly Income (after taxes) \\
\$$1,983.05$ & Money Left Over For Other Stuff \\
\hline
\hline
\end{tabular}
\caption{The final budget.}
\label{table:budget}
\end{center}
\end{table}
By scaling the insurance 
payments up into pre-tax dollars before making the estimate, 
the bills come in under the $35$\% debt ratio.

\section{Summary}

I hope this tool is as useful to you as it has been for me.  
It's been fun to learn Java and provide this tool as a free 
application.

\end{document}
